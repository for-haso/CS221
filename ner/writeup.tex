\documentclass[12pt]{article}
\usepackage{fullpage,enumitem,amsmath,amssymb,graphicx}

\begin{document}

\begin{center}
{\Large CS221 Fall 2013 ner}

\begin{tabular}{rl}
SUNet ID: & nisham \\
Name: & Nisha Masharani \\
Collaborators: & 
\end{tabular}
\end{center}

By turning in this assignment, I agree by the Stanford honor code and declare
that all of this is my own work.

\section*{Problem 3}

\begin{enumerate}[label=(\alph*)]
  \item The treewidth is 2, because both $y_5$ and $y_10$ have potentials with $y_4$, so when $y_4$ is eliminated, the potential created has arity 2. All other potentials created by variable elimination in this case have arity 1.\\
  The variable elimination order eliminates the variables with the fewest neighbors first. So, $y_1$ through $y_3$ will be eliminated, then $y_5$ through $y_9$ will be eliminated, then $y_10$ will be eliminated, then $y_4$ and $y_10$ will be eliminated.
  \item 
  \begin{eqnarray*}
  p(y | x; \theta) &=& \frac{1}{Z(x;\theta)} \prod_{i = 1}^T G_i(y_{i-1}, y_i| x; \theta)\\
  p(y_t | y_{-t}, x; \theta) &=& \frac{p(y_t, y_{-t} | x; \theta)}{p(y_{-t}| x; \theta)}\\
  &=& \frac{p(y | x; \theta)}{p(y_{-t}| x; \theta)}\\
  &=& \frac{\frac{1}{Z(x;\theta)} \prod_{i = 1}^T G_i(y_{i-1}, y_i| x; \theta)}{\frac{1}{Z(x;\theta)} \prod_{i = 1, i \notin \{t, t+1\}}^T G_i(y_{i-1}, y_i| x; \theta)}\\
  &=& G_t(y_{t-1}, y_t | x; \theta) G_{t+1}(y_t, y_{t+1} | x; \theta)\\
  \end{eqnarray*}
  Edge case: if $t = T$, then $p(y_t | y_{-t}, x; \theta) = G_t(y_{t-1}, y_t | x; \theta)$. Similarly, if $t = 1$, then $y_{t-1} = BEGIN\_TAG$ .\\
\end{enumerate}

\end{document}
