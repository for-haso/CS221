\documentclass[12pt]{article}
\usepackage{fullpage,enumitem,amsmath,amssymb,graphicx}

\begin{document}

\begin{center}
{\Large CS221 Fall 2013 Homework 1: Warmup}

\begin{tabular}{rl}
SUNet ID: & nisham \\
Name: & Nisha Masharani \\
Collaborators: & [list all the people you worked with]
\end{tabular}
\end{center}

By turning in this assignment, I agree by the Stanford honor code and declare
that all of this is my own work.

\section*{Problem 1}

\begin{enumerate}[label=(\alph*)]

  \item To minimize, solve $f'(x) = 0$.
  \begin{eqnarray*}
  f(x) &=& \frac{1}{2}\displaystyle\sum\limits_{i=1}^n w_i (x-b_i)^2\\
  f'(x) &=& \frac{d}{dx}\biggl[\frac{1}{2}\displaystyle\sum\limits_{i=1}^n w_i (x-b_i)^2\biggr]\\
  &=& \frac{1}{2}\displaystyle\sum\limits_{i=1}^n 2w_i (x-b_i)\\
  &=& \displaystyle\sum\limits_{i=1}^n w_i (x-b_i)\\
  f'(x) &=& 0\\
  \displaystyle\sum\limits_{i=1}^n w_i (x-b_i) &=& 0\\
  \displaystyle\sum\limits_{i=1}^n w_i x - \displaystyle\sum\limits_{i=1}^n w_i b_i &=& 0\\
  x \displaystyle\sum\limits_{i=1}^n w_i &=& \displaystyle\sum\limits_{i=1}^n w_i b_i\\
  x &=& \frac{\sum_{i=1}^n w_i b_i}{\sum_{i=1}^n w_i}
  \end{eqnarray*}

  \item 
  \begin{eqnarray*}
  f(x) &=& \max_{a \in \lbrace1, -1\rbrace} \displaystyle\sum\limits_{j=1}^d ax_j\\
  &=& max \biggl\lbrace \displaystyle\sum\limits_{j=1}^d x_j , \displaystyle\sum\limits_{j=1}^d -x_j \biggr\rbrace\\
  g(x) &=& \displaystyle\sum\limits_{j=1}^d \max_{a \in \{1,-1\}} a x_j\\
  &=& \displaystyle\sum\limits_{j=1}^d \max \biggl\lbrace x_j , -x_j \biggr\rbrace\\
  \end{eqnarray*}
  If all $x_j \ge 0$ or all $x_j \le 0$, then $f(x) = g(x)$ because $f(x)$ will take the sum of all $x_j$ and then take the absolute value to find the max, while $g(x)$ will take the absolute value of all $x_j$ and then sum them, resulting in the same answer if the signs of all $x_j$ are the same. If the signs of all $x_j$ are not the same, then $g(x) > f(x)$ because at least one element in the sum of $f(x)$ will be different from the others, so adding that number to the sum will decrease the magnitude of the sum. Thus, $g(x) \ge f(x)$.

  \item
  $X_i = \left \{
     \begin{array}{ll}
       0 & \mbox{if the ith roll is 1, 2, or 3}\\
       4 & \mbox{if the ith roll is 4}\\
       5 & \mbox{if the ith roll is 5}\\
       6 & \mbox{if the ith roll is 6}\\
     \end{array}
   \right.
   $
  \begin{eqnarray*}
  E\biggl[\displaystyle\sum\limits_{i=1}^n X_i\biggr] &=& \displaystyle\sum\limits_{i=1}^n E\biggl[X_i\biggr]\\
  E\biggl[X_i\biggr] &=& \frac{3}{6} (0) + \frac{1}{6} (4) + \frac{1}{6} (5) + \frac{1}{6} (6) \\
  &=& \frac{15}{6}\\
  E\biggl[\displaystyle\sum\limits_{i=1}^n X_i\biggr] &=& \displaystyle\sum\limits_{i=1}^n E\biggl[X_i\biggr]\\
  &=& \frac{15}{6} n
  \end{eqnarray*}
  The dice rolls do not have to be independent for the expectation of the sum to equal the sum of the expectation.

  \item
  $L(p) = p^3(1-p)^2$\\
  If we maximize $\log(L(p))$, that is the same as maximizing $L(p)$, because the log function is one to one and increasing, so the $p$ value at the maxima of both functions are the same.
  \begin{eqnarray*}
  \log(L(p)) &=& \log(p^3(1-p)^2)\\
  &=& \log(p^3) + \log((1-p)^2)\\
  &=& 3\log(p) + 2\log(1-p)\\
  \frac{d}{dp}\biggl[\log(L(p))\biggr] &=& 3\biggl(\frac{1}{p}\biggr) + -2\biggl(\frac{1}{1-p}\biggr)\\
  &=& \frac{3(1-p) - 2(p)}{p(1-p)}\\
  &=& 0\\
  \frac{3(1-p) - 2(p)}{p(1-p)} &=& 0\\
  3(1-p) - 2(p) &=& 0\\
  3 - 3p &=& 2p\\
  p &=& \frac{3}{5}\\
  \end{eqnarray*}

\end{enumerate}

\section*{Problem 2}

\begin{enumerate}[label=(\alph*)]
  \item For each word in the sentence, there are 4 classifications. There are n words in the sentence. Thus, there are $4^n$ possible different tag sequences.
  
  \item For each rectangle, we want to define dimensions and a position. We can do this in several ways, but the easiest way to do so is to use the grid lines in the pixel grid. We need two horizontal lines and two vertical lines to define a grid. In an $n x n$ grid, there are $n + 1$ vertical edges and $n + 1$ horizontal edges. Since we're choosing two of each, the number of possibilities for each rectangle is:
  \begin{eqnarray*}
  \binom{n+1}{2}\binom{n+1}{2} &=& \frac{(n+1)!(n+1)!}{(n-1)!2!(n-1)!2!}\\
  &=& \frac{(n+1)n(n+1)n}{4}\\
  \end{eqnarray*}
  Thus, the asymptotic complexity is $O(n^4)$

  \item TODO(nisham)
\end{enumerate}

\end{document}
