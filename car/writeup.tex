\documentclass[12pt]{article}
\usepackage{fullpage,enumitem,amsmath,amssymb,graphicx}

\begin{document}

\begin{center}
{\Large CS221 Fall 2013 Homework 8: car}

\begin{tabular}{rl}
SUNet ID: & nisham \\
Name: & Nisha Masharani \\
Collaborators: & esclee
\end{tabular}
\end{center}

By turning in this assignment, I agree by the Stanford honor code and declare
that all of this is my own work.

\section*{Problem 4}

\begin{enumerate}[label=(\alph*)]
  \item 
  \begin{eqnarray*}
  p(c_{11}, c_{12} | e_1) &\propto& p(c_{11}, c_{12})p(e_1|c_{11}, c_{12})\\
  &\propto& p(c_{11})p(c_{12})p(e_1|c_{11}, c_{12})\\
  p(e_1|c_{11}, c_{12}) &=& \frac{1}{2}\left(p_{\mathcal N}(e_{11}; ||a_1 - c_{11}||, \sigma^2)p_{\mathcal N}(e_{12}; ||a_1 - c_{12}||, \sigma^2)\right) \\
  &+& \frac{1}{2}\left(p_{\mathcal N}(e_{12}; ||a_1 - c_{11}||, \sigma^2)p_{\mathcal N}(e_{11}; ||a_1 - c_{12}||, \sigma^2)\right)\\
  p(c_{11}, c_{12} | e_1) &\propto& p(c_{11})p(c_{12})\left[\frac{1}{2}\left(p_{\mathcal N}(e_{11}; ||a_1 - c_{11}||, \sigma^2)p_{\mathcal N}(e_{12}; ||a_1 - c_{12}||, \sigma^2)\right)\right. \\
  &+& \left.\frac{1}{2}\left(p_{\mathcal N}(e_{12}; ||a_1 - c_{11}||, \sigma^2)p_{\mathcal N}(e_{11}; ||a_1 - c_{12}||, \sigma^2)\right)\right]
  \end{eqnarray*}
  \item If we want to maximize $p(c_{11}, \dots, c_{1K} \mid e_1)$, there is at least one set of locations $c_{11}, \dots, c_{1K}$ that will maximize this value by having all $c_{1i}$ such that the distance between $c_{1i}$ and $a_1$ uniquely is close to a $e_{1,j}$ in $e_1$. Since the probability for each car i to be at any location is constant, the cars at these optimal locations are interchangable. Therefore, we have $k$ locations, and $k$ cars, so there are $k!$ ways to put each of the cars into a location.
  \item At each timestep $t$, each observed distance $e_{ti}$ depends on the car locations $c_{t1}, ..., c_{tk}$, so there is a factor between each $e_{ti}$ and all $c_{t1}, ..., c_{tk}$. Alternately, we can think that all car locations $c_{t1}, ..., c_{tk}$ depends on each $e_{ti}$. If we condition on the values of $e_{t1}, ..., e_{tk}$, then we have that all car locations $c_{t1}, ..., c_{tk}$ are linked by $k$ factors that previously joined all car locations to each $e_{ti}$. Each of these $k$ factors has an arity of $k$. Each car location also depends on the previous car location, a factor which is not changed by eliminating $e_{t1}, ..., e_{tk}$. Therefore, the maximum arity after eliminating all $e_{t1}, ..., e_{tk}$ is $k$, so the tree width is $k$.
  \end{enumerate}

\end{document}
